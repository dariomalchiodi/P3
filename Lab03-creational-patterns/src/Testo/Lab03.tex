\documentclass{article}
%\usepackage{t1enc}
%\usepackage[latin1]{inputenc}
\usepackage[italian]{babel}
%\usepackage[subscriptcorrection,mtbold]{mathtime}
\usepackage{mathrsfs}
\usepackage{alltt,theorem,xspace}
\usepackage{bm,amsmath,amssymb}

\advance\textwidth by 4cm
\advance\oddsidemargin by -2cm
\advance\evensidemargin by -2cm

\advance\textheight by 2cm
\advance\topmargin by -1cm


\title{Sistemi per l'Elaborazione dell'Informazione 2}
\date{}

\begin{document}

\maketitle

\section*{Intervalli}

Lo scopo di questa esercitazione \`e la creazione di tre implementazioni di una
classe che rappresenta intervalli di interi. Ogni intervallo � definito da due
estremi interi, quello sinistro e quello destro, che sono inclusi
nell'intervallo. In tutte le implementazioni le istanze sono immutabili.


\subsection*{Prima implementazione (elementare)}

La prima implementazione, che chiameremo \texttt{BasicInterval}, ha attributi \texttt{left} e \texttt{right} pubblici e finali, e due costruttori, uno che chiede entrambi gli intervalli mentre l'altro chiede solo un punto. Il metodo \texttt{length()} deve fornire il
numero di interi contenuto nell'intervallo. Il metodo \texttt{contains(int)}
deve restituire \texttt{true} se l'argomento cade nell'intervallo,
\texttt{false} altrimenti. Ricordatevi di implementare ragionevolmente
\texttt{equals()} e \texttt{hashCode()} (potete ad esempio fare riferimento al terzo capitolo di J. Bloch, \emph{Effective Java}, disponibile on-line all'indirizzo \texttt{http://java.sun.com/developer/Books/effectivejava/Chapter3.pdf}).

\`E consigliabile creare un'opportuna interfaccia \texttt{Interval} che verr\`a implementata da tutte le classi rappresentanti intervalli che andremo a realizzare durante questa esercitazione (vi semplificher\`a molto la risoluzione dell'ultimo esercizio); \`e anche consigliabile dividere implementazioni differenti in package differenti.


\subsection*{Seconda implementazione (con metodi di fabbricazione e pesi piuma
  per gli intervalli di lunghezza uno)}

Nella seconda implementazione, che chiameremo \texttt{EnhancedInterval}, i costruttori sono privati e sostituiti da due
metodi di fabbricazione. La classe costruisce all'inizializzazione tutti gli
intervalli singoletto nonnegativi fino a una costante \texttt{MAX\_FLYWEIGHT}, e
li memorizza in un vettore. I metodi di fabbricazione devono restituire le
copie memorizzate ogni volta che questo \`e possibile.

\subsection*{Terza implementazione (con metodi di fabbricazione che
  restituiscono sottoclassi specifiche, pesi piuma per gli intervalli
  di lunghezza uno e un singoletto per l'intervallo vuoto)}

La terza implementazione spinge al massimo l'ottimizzazione dello spazio. Per
ottenere questo scopo, aumentiamo l'astrazione: gli estremi vengono ottenuti
tramite i metodi \texttt{left()} e \texttt{right()}. La classe principale, che chiameremo \texttt{AbstractInterval}, diventa ora astratta.

Vogliamo inoltre avere la possibilit\`a di rappresentare l'intervallo
vuoto. L'intervallo vuoto ha lunghezza zero, non contiene nessun intero e
lancia un'eccezione \texttt{UnsupportedOperationException} se si cercano di
sapere i suoi estremi.

La classe \texttt{AbstractInterval} implementa 
solo alcuni dei metodi (\texttt{length()} e \texttt{contains(int)} rappresentano ovvi esempi, ma potete anche ragionare su \texttt{equals()}), lasciando i rimanenti astratti. A questo punto, ci sono tre sottoclassi concrete di
\texttt{AbstractInterval}:
\begin{enumerate}
\item \texttt{EmptyInterval}, privo di attributi, che rappresenta l'intervallo
  vuoto. La classe va implementata come un singoletto (non ha senso avere pi\`u
  intervalli vuoti).
\item \texttt{PointInterval}, con un solo attributo \texttt{point}, che
  rappresenta gli intervalli di lunghezza uno. Questa classe deve occuparsi
  anche della gestione dei pesi piuma.
\item \texttt{FullInterval}, con attributi \texttt{left} e \texttt{right}, che
  rappresenta gli intervalli rimanenti.
\end{enumerate}
Tutte le classi devono ovviamente implementare \texttt{left()} e
\texttt{right()}. Inoltre, devono fornire dei metodi di fabbricazione con
visibilit\`a limitata al pacchetto (opzionalmente, \texttt{EmptyInterval} pu\`o
rendere accessibile la sua sola istanza in un campo statico finale con
visibilit\`a di pacchetto inizializzato direttamente). Notate che ci vuole un po'
di cura nell'implementazione di \texttt{equals()} e \texttt{hashCode()}.

Infine, \texttt{Interval} conterr\`a i metodi (statici) di
fabbricazione generici per gli intervalli.
Potete scegliere i metodi in vario modo, ad esempio
chiamandoli \texttt{getInstance(int,int)}, \texttt{getInstance(int)} e
\texttt{getInstance()} (per l'intervallo vuoto), oppure in modo pi\`u creativo, come 
\texttt{getInterval(int,int)}, \texttt{getPointInterval(int)} e
\texttt{getEmptyInterval()}. I metodi di fabbricazione restituiscono sempre un
\texttt{Interval}.

\subsection*{Quarta implementazione (tramite fabbrica astratta che
contiene metodi di fabbricazione, e pu\`o venire istanziata
in almeno due forme)}

La quarta implementazione cerca di essere, per quanto possibile, flessibile ed
efficiente, lasciando all'utente la scelta dell'implementazione. Dovete dichiarare
una fabbrica astratta di intervalli (cio\`e un'interfaccia) che contiene i metodi di fabbricazione.
Dovete quindi implementare almeno due diverse fabbriche concrete (per esempio, con e senza
pesi piuma, ma se avete seguito i suggerimenti dati alla fine del primo esercizio dovreste riuscire a riutilizzare le tre implementazioni fatte finora). Le fabbriche concrete devono essere restituite da un metodo \texttt{getFactory(String)}, da inserire in una utility (cio\`e una classe che contiene esclusivamente metodi e attributi statici, come per esempio accade con \texttt{Math}),
a cui viene fornito un nome simbolico (che dovreste  mettere in una stringa pubblica, statica e finale). Verificate anche se abbia senso oppure no implementare le fabbriche concrete come singoletti.

Per verificare che tutto avviene come si deve, scrivete un semplice \texttt{main()} che
chiede all'utente il nome di una fabbrica, la istanza, crea i tre tipi fondamentali di intervalli e
ne stampa la classe implementante (usate \texttt{getClass()} sugli oggetti restituiti dalla fabbrica),
in modo da controllare che la creazione degli oggetti sia andata come vi aspettavate.

\subsection*{Parte facoltativa}
Utilizzando UML scrivete i diagrammi delle classi  e i diagrammi di sequenza per tutti gli esercizi risolti. Scrivete la documentazione JavaDoc per tutte le classi e i
metodi coinvolti, evidenziando gli invarianti. Infine, scrivete dei test di
funzionamento per tutte le classi coinvolte.

\end{document}
