\documentclass{article}
%\usepackage{t1enc}
\usepackage[latin1]{inputenc}
\usepackage[italian]{babel}
%\usepackage[subscriptcorrection,mtbold]{mathtime}
\usepackage{mathrsfs}
\usepackage{alltt,theorem,xspace}
\usepackage{bm,amsmath,amssymb}

\advance\textwidth by 4cm
\advance\oddsidemargin by -2cm
\advance\evensidemargin by -2cm

\advance\textheight by 2cm
\advance\topmargin by -1cm

\title{Sistemi per l'elaborazione dell'informazione 2}
\date{}

\begin{document}
\shorthandoff{"}

\maketitle

\section*{Serializzazione logica}

A volte il formato standard di serializzazione generato da Java non \`e il pi\`u adatto. Questo \`e vero, in particolare, se lo stato interno di un oggetto presenta ridondanza.

Per prendere coscienza del problema, provate a estendere \texttt{java.util.HashMap} in modo che fornisca un servizio di accesso invertito: di solito, una \texttt{Map} permette, tramite il metodo \texttt{get()}, di ottenere il valore associato a una chiave. Dovete aggiungere un metodo \texttt{lookup()} che permetta di ottenere una chiave, dato un valore. Per fare questo, estendete \texttt{HashMap} in modo che contenga una \emph{seconda} mappa in cui inserite le coppie chiave/valore in ordine invertito (per ottenere questo effetto \`e necessario sovrascrivere \texttt{put()} e \texttt{clear()}). In pratica, state implementando una biiezione, e dovete fare in modo che le invocazioni di \texttt{put()} non violino le propriet\`a matematiche corrispondenti. Potete lanciare una \texttt{IllegalStateException} per segnalare una condizione di errore.

Ovviamente, se rendete la nuova classe serializzabile verranno salvate inutilmente \emph{due} copie delle coppie chiave/valore: una nella \texttt{HashMap} che state estendendo, e una nella mappa di inversione. Per ovviare a questo inconveniente, dovete rendere la mappa di inversione transiente, e modificare \texttt{readObject()} in modo che la ricostruisca enumerando le coppie chiave/valore della mappa principale (dovete utilizzare il metodo \texttt{entrySet()} per enumerare le coppie, e reinserirle in una nuova mappa di inversione con chiave e valore scambiati).

Realizzate un metodo \texttt{main()} o una classe di test in cui istanziate la classe che avete creato, la popolate con alcune associazioni, ne stampate il contenuto (utilizzando un metodo \texttt{toString()} opportunamente creato), serializzate l'oggetto ottenuto, lo deserializzate e stampate il contenuto dell'oggetto che ottenete, verificando che gli output coincidono.

Create infine un'applicazione che visualizzi una biiezione all'interno di una finestra contenente i seguenti widget:

\begin{itemize}
\item una lista che visualizzi le coppie (chiave, valore) contenute nella biiezione
\item un pulsante che permetta di rimuovere dalla biiezione le coppie (chiave, valore) selezionate nella lista;
\item un pulsante che permetta di rimuovere dalla biiezione tutte le coppie in essa contenute;
\item due caselle di testo deputate a contenere una chiave e un valore, affiancate a un pulstante che permetta di inserire una nuova coppia nella biiezione, utilizzando i valori inseriti nelle caselle di testo.
\end{itemize}

Scegliete un opportuno dispositore per inserire i widget nella shell e gestite la segnalazione di errore nel caso in cui l'inserimento di una nuova coppia violi l'integrit\`a della biiezione (usando magari una \texttt{MessageBox}). Aggiungete poi alla vostra applicazione un menu che permetta di serializzare e deserializzare la biiezione (quando deserializzate eliminate l'istanza precedente di \texttt{BijectiveHashMap}). I pi\`u arditi possono anche verificare i casi specifici in cui si carica una nuova biiezione senza aver salvato quella corrente, avvertendo l'utente della potenziale perdita di dati e offrendo la possibilit\`a di salvare la biiezione corrente prima di procedere. In tal caso sar\`a utile utilizzare il  disegno \emph{command} e un \texttt{MessageBox} opportuno.

\section*{\texttt{readResolve()}}

A volte non \`e possibile utilizzare in assoluto l'oggetto generato implicitamente dal sistema di serializzazione. (Un tipico caso \`e quello dei singoletti che vedremo in una delle prossime lezioni).

Per risolvere questi problemi, Java fornisce il metodo \texttt{readResolve()} (oltre alla sua controparte \texttt{writeReplace()}, che per\`o non vi servir\`a in questo esercizio). Se viene definito un metodo
\begin{verbatim}
private Object readResolve()
\end{verbatim}
all'interno di una classe, esso verr\`a invocato al momento della deserializzazione (e \texttt{this} sar\`a l'oggetto deserializzato). Il metodo restituisce un oggetto che sar\`a considerato il risultato della deserializzazione. Per esempio, il metodo pu\`o decidere di buttare via la nuova istanza (cio\`e \texttt{this}) e restituire un altro oggetto.


Riscrivete la soluzione del primo esercizio utilizzando il metodo \texttt{readResolve}.

\end{document}

